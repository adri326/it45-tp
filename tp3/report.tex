\documentclass[12pt]{article}

\setlength{\parskip}{1em}

\usepackage[T1]{fontenc}
\usepackage[a4paper, margin=0.7in]{geometry}
\usepackage{amsfonts}
\usepackage{amsmath}
\usepackage{amssymb}
\usepackage{stmaryrd}
\usepackage{xcolor}
\usepackage{subcaption}
\usepackage{graphicx}
\usepackage{float}
\usepackage{pdfpages}
\usepackage[french]{babel}
\usepackage{listings}
\usepackage{hyperref}
\usepackage{tikz}
\usepackage{pgfplots}
\usepackage{charter}

\pgfplotsset{compat=1.18}

% \usepgfplotslibrary{external}
% \tikzexternalize

\hypersetup{
  colorlinks=true,
  linkcolor=blue,
  urlcolor=blue,
  pdftitle={RS40: TP1}
}

\lstset{literate=
  {á}{{\'a}}1 {é}{{\'e}}1 {í}{{\'i}}1 {ó}{{\'o}}1 {ú}{{\'u}}1
  {Á}{{\'A}}1 {É}{{\'E}}1 {Í}{{\'I}}1 {Ó}{{\'O}}1 {Ú}{{\'U}}1
  {à}{{\`a}}1 {è}{{\`e}}1 {ì}{{\`i}}1 {ò}{{\`o}}1 {ù}{{\`u}}1
  {À}{{\`A}}1 {È}{{\'E}}1 {Ì}{{\`I}}1 {Ò}{{\`O}}1 {Ù}{{\`U}}1
  {ä}{{\"a}}1 {ë}{{\"e}}1 {ï}{{\"i}}1 {ö}{{\"o}}1 {ü}{{\"u}}1
  {Ä}{{\"A}}1 {Ë}{{\"E}}1 {Ï}{{\"I}}1 {Ö}{{\"O}}1 {Ü}{{\"U}}1
  {â}{{\^a}}1 {ê}{{\^e}}1 {î}{{\^i}}1 {ô}{{\^o}}1 {û}{{\^u}}1
  {Â}{{\^A}}1 {Ê}{{\^E}}1 {Î}{{\^I}}1 {Ô}{{\^O}}1 {Û}{{\^U}}1
  {ã}{{\~a}}1 {ẽ}{{\~e}}1 {ĩ}{{\~i}}1 {õ}{{\~o}}1 {ũ}{{\~u}}1
  {Ã}{{\~A}}1 {Ẽ}{{\~E}}1 {Ĩ}{{\~I}}1 {Õ}{{\~O}}1 {Ũ}{{\~U}}1
  {œ}{{\oe}}1 {Œ}{{\OE}}1 {æ}{{\ae}}1 {Æ}{{\AE}}1 {ß}{{\ss}}1
  {ű}{{\H{u}}}1 {Ű}{{\H{U}}}1 {ő}{{\H{o}}}1 {Ő}{{\H{O}}}1
  {ç}{{\c c}}1 {Ç}{{\c C}}1 {ø}{{\o}}1 {å}{{\r a}}1 {Å}{{\r A}}1
  {€}{{\euro}}1 {£}{{\pounds}}1 {«}{{\guillemotleft}}1
  {»}{{\guillemotright}}1 {ñ}{{\~n}}1 {Ñ}{{\~N}}1 {¿}{{?`}}1 {¡}{{!`}}1
  {Σ}{{$\Sigma$}}1 {≠}{{$\neq$}}1
}

\newcommand{\concat}{\ensuremath{\mbox{+\!\!\!+}\,}}

\definecolor{bgColor}{rgb}{0.97, 0.97, 0.965}
\definecolor{darkcyan}{rgb}{0.254, 0.545, 0.505}

\lstdefinestyle{C}{
  aboveskip=0.2cm,
  belowskip=0.2cm,
  backgroundcolor=\color{bgColor},
  commentstyle=\color{gray},
  keywordstyle=\color{magenta},
  numberstyle=\tiny\color{gray},
  stringstyle=\color{purple},
  basicstyle=\footnotesize,
  breakatwhitespace=false,
  breaklines=true,
  captionpos=b,
  keepspaces=true,
  numbers=left,
  numbersep=5pt,
  showspaces=false,
  showstringspaces=false,
  showtabs=false,
  tabsize=2,
  language=C
}

\title{Rapport de TP3 d'IT45}
\author{Adrien Burgun}
\date{Printemps 2022}

\begin{document}

\maketitle

\section{Implémentation de l'algorithme de Little}

\subsection{Calcul des distances}

Nous avons en entrée les variables $N_{cities}$, le nombre de villes, et $(x_i, y_i)$, les coordonnées de celles-cis.
Pour pouvoir utiliser l'algorithme de Little, nous calculons la matrices des distances $D$, tel que $D_{ij}$ soit la distance de la ville $i$ jusqu'à la ville $j$.
Nous pouvons utiliser la relation

\[
  \forall i \neq j, \quad D_{ij} = \sqrt{(x_i - x_j)^2 + (y_i - y_j)^2}
\]

La signature dans l'implémentation en C de la matrice $D$ diffère de celle initiallement proposée: je voulais que le code ne soit pas dépendant de la variable \texttt{NBR\_TOWNS}. Il m'est donc impossible d'indiquer au compileur la taille de $D$ sans utiliser des VLA.
Pour accéder à $D_{ij}$, il faut donc écrire \texttt{dist[j * n\_cities + i]} au lieu de \texttt{dist[j][i]}.

\lstinputlisting[style=C, firstline=95, lastline=107]{little.c}

\subsection{Heuristique du voisin le plus proche}

Pour que l'algorithme de Little fonctionne au mieux, il faut lui donner comme information une borne initiale de la distance parcourue.
Plus cette borne est proche de la solution optimale, plus l'algorithme de Little sera efficace à ignorer les branches non-optimales.

Nous construisons cette solution initiale en suivant l'heuristique du voisin le plus proche.
À l'étape $i$, on choisit la prochaine ville de la manière suivante:

\begin{align*}
  u_{i+1} &= \arg \min_{k \in S_i}(D[u_i,u_k]) \\
  S_i &= \llbracket 0; \; N_{cities} \rrbracket \; \backslash \; \{ k \;|\; \exists \; n \le i, u_n = k \}
\end{align*}

\lstinputlisting[style=C, linerange={162-200,211-222}]{little.c}

\subsection{Algorithme de Little}

L'algorithme de Little cherche le cycle Hamiltonien optimal en appliquant la méthode de Branch and Bound avec les informations que l'on peut déduire de la matrice des distances $D$.

À chaque noeud, on vérifie si l'on a construit un chemin entier et on compare dans ce cas ce chemin avec la meilleure solution trouvée auparavant:

\lstinputlisting[style=C, linerange={478-483,485-490,495-495}]{little.c}

Ensuite, on soustrait d'abords à chaque colonne $i$ la valeur de $\min_{j}(D_{ij})$, et on incrémente la valeur minimale du noeud (\texttt{eval}) par cette quantité.
On fait de même pour les lignes $j$; on obtient alors $D'$, une matrice où chaque colonne et chaque ligne est soit infinie, soit contient un ou plusieurs zéros.

\lstinputlisting[style=C, linerange={507-519,521-539,541-545}]{little.c}

On vérifie ensuite si la nouvelle valeur de \texttt{eval} ne dépasse pas la valeur optimale trouvée \texttt{best\_eval}; si c'est le cas, on arrête la recherche du noeud (opération \og Cut \fg):

\lstinputlisting[style=C, linerange={547-548,552-553}]{little.c}

Ensuite, on cherche dans la matrice $D'$ le zéro ayant la plus grande pénalité:

\begin{align*}
  (i^{\star}, j^{\star}) &= \arg \max_{i,j}(P_{ij} \; | \; D'_{ij} = 0 \; \land \; P_{ij} < \infty) \\
  P_{ij} &= \min_{k \neq i}(D'_{kj}) + \min_{k \neq j}(D'_{ik})
\end{align*}

\lstinputlisting[style=C, linerange={561-584}]{little.c}

Enfin, si $(i^{\star}, j^{\star})$ existent, on explore la branche $(S \concat \{i^{\star} \rightarrow j^{\star}\}, \texttt{eval})$ et la branche $(S \concat \{i^{\star} \nrightarrow j^{\star}\}, \texttt{eval} + P_{i^{\star}j^{\star}})$:

\lstinputlisting[style=C, linerange={587-599,601-602,607-615,619-621,630-633,637-638}]{little.c}

\subsection{Optimisations}

L'ennemi numéro 1 aux performances sont les allocations de mémoire.
Comme on ne connait pas à l'avance le nombre de villes dans le problème, il faut allouer la mémoire nécessaire avant l'appel à la fonction récursive, puis libérer cette mémoire après cet appel.

Pour ce faire, l'algorithme de Little est séparé en deux fonctions: \texttt{little\_algorithm}, qui gère cette mémoire, et \texttt{little\_algorithm\_rec}, qui est appelé par ce premier.

Chaque étape de l'algorithme de l'algorithme de Little prend en entrée une matrice $D$ et donne en sortie deux matrices:

\begin{itemize}
  \item $D'$, pour la branche $i^{\star} \rightarrow j^{\star}$, où la colonne $i^{\star}$ et la ligne $j^{\star}$ sont remplis avec des $\infty$ et $D'[j^{\star}, i^{\star}] := \infty$
  \item $D''$, pour la branche $i^{\star} \nrightarrow j^{\star}$, où $D''[i^{\star}, j^{\star}] := \infty$
\end{itemize}

Comme dans la branche $i^{\star} \rightarrow j^{\star}$, le compteur \texttt{iteration} est incrémenté, et dans la branche $i^{\star} \nrightarrow j^{\star}$, ce compteur reste le même, on peut alors placer $D$, $D'$ et $D''$ dans une liste de $N_{cities} + 1$ matrices de $N_{cities} \times N_{cities}$, de la manière suivante:

\begin{itemize}
  \item $D$ est lue à l'indice \texttt{iteration}
  \item $D'$ est construite en copiant $D$ à l'indice $\texttt{iteration}+1$ et en le modifiant
  \item $D''$ est construite en modifiant $D$ directement à l'indice \texttt{iteration}, après avoir créé la copie pour $D'$
\end{itemize}

Les seules autres allocations à optimiser sont celles utilisées par les fonctions de vérification de la validité du chemin.

Le code est enfin compilé avec \texttt{-O3} pour utiliser les opérations de calcul vectoriel de la CPU si possible.

\section{Résultats}

Durant le développement du code, des tests automatisés ont été écrits en parallèle dans le fichier \texttt{test.c}; ceux-cis vérifient le bon fonctionnement du code sur des cas à part et des cas aléatoires.

Pour 6 villes, la solution optimale trouvée est $0, 4, 5, 3, 2, 1$.
Comme la matrice des distances est symmétrique, on peut transformer la solution tel que $S[1] < S[n-1]$.
On obtient alors $0, 1, 2, 3, 5, 4$, avec une distance de $2315.15$.

Pour 10 villes, on obtient $0, 1, 6, 2, 7, 8, 9, 3, 5, 4$ et une distance de $2826.50$.

Pour les 52 villes, la solution optimale a une distance de $7544.37$:

% 00 48 31 44 18 40 07 08 09 42 32 50 10 51 13 12 46 25 26 27 11 24 03 05 14 04 23 47 37 36 39 38 35 34 33 43 45 15 28 49 19 22 29 01 06 41 20 16 02 17 30 21

\begin{figure}[H]
  \caption{Chemin tracé par la solution optimale de Berlin52}
  \center
  \begin{tikzpicture}
    \node[circle,fill,inner sep=1pt] at (3.25, 3.30) (1) {};
    \node[circle,fill,inner sep=1pt] at (0.14, 1.06) (2) {};
    \node[circle,fill,inner sep=1pt] at (1.98, 4.31) (3) {};
    \node[circle,fill,inner sep=1pt] at (5.43, 3.94) (4) {};
    \node[circle,fill,inner sep=1pt] at (4.86, 3.76) (5) {};
    \node[circle,fill,inner sep=1pt] at (5.06, 3.79) (6) {};
    \node[circle,fill,inner sep=1pt] at (0.14, 1.32) (7) {};
    \node[circle,fill,inner sep=1pt] at (3.02, 5.75) (8) {};
    \node[circle,fill,inner sep=1pt] at (3.33, 6.75) (9) {};
    \node[circle,fill,inner sep=1pt] at (3.74, 6.49) (10) {};
    \node[circle,fill,inner sep=1pt] at (9.22, 3.56) (11) {};
    \node[circle,fill,inner sep=1pt] at (7.01, 3.33) (12) {};
    \node[circle,fill,inner sep=1pt] at (8.42, 1.15) (13) {};
    \node[circle,fill,inner sep=1pt] at (8.79, 0.03) (14) {};
    \node[circle,fill,inner sep=1pt] at (4.86, 3.91) (15) {};
    \node[circle,fill,inner sep=1pt] at (4.17, 2.13) (16) {};
    \node[circle,fill,inner sep=1pt] at (0.83, 3.82) (17) {};
    \node[circle,fill,inner sep=1pt] at (2.39, 3.65) (18) {};
    \node[circle,fill,inner sep=1pt] at (2.93, 5.03) (19) {};
    \node[circle,fill,inner sep=1pt] at (3.22, 2.10) (20) {};
    \node[circle,fill,inner sep=1pt] at (1.72, 2.67) (21) {};
    \node[circle,fill,inner sep=1pt] at (2.99, 3.36) (22) {};
    \node[circle,fill,inner sep=1pt] at (2.76, 2.39) (23) {};
    \node[circle,fill,inner sep=1pt] at (4.80, 3.59) (24) {};
    \node[circle,fill,inner sep=1pt] at (5.60, 3.33) (25) {};
    \node[circle,fill,inner sep=1pt] at (6.98, 1.41) (26) {};
    \node[circle,fill,inner sep=1pt] at (7.59, 1.81) (27) {};
    \node[circle,fill,inner sep=1pt] at (7.18, 2.30) (28) {};
    \node[circle,fill,inner sep=1pt] at (3.79, 1.03) (29) {};
    \node[circle,fill,inner sep=1pt] at (2.36, 1.44) (30) {};
    \node[circle,fill,inner sep=1pt] at (2.41, 3.19) (31) {};
    \node[circle,fill,inner sep=1pt] at (3.30, 3.82) (32) {};
    \node[circle,fill,inner sep=1pt] at (6.61, 6.67) (33) {};
    \node[circle,fill,inner sep=1pt] at (4.02, 3.33) (34) {};
    \node[circle,fill,inner sep=1pt] at (3.94, 3.42) (35) {};
    \node[circle,fill,inner sep=1pt] at (3.94, 3.51) (36) {};
    \node[circle,fill,inner sep=1pt] at (4.43, 3.51) (37) {};
    \node[circle,fill,inner sep=1pt] at (4.57, 3.71) (38) {};
    \node[circle,fill,inner sep=1pt] at (4.14, 3.65) (39) {};
    \node[circle,fill,inner sep=1pt] at (4.37, 3.74) (40) {};
    \node[circle,fill,inner sep=1pt] at (2.73, 5.52) (41) {};
    \node[circle,fill,inner sep=1pt] at (0.55, 1.49) (42) {};
    \node[circle,fill,inner sep=1pt] at (5.03, 5.29) (43) {};
    \node[circle,fill,inner sep=1pt] at (4.02, 2.87) (44) {};
    \node[circle,fill,inner sep=1pt] at (3.19, 4.68) (45) {};
    \node[circle,fill,inner sep=1pt] at (4.77, 2.79) (46) {};
    \node[circle,fill,inner sep=1pt] at (6.72, 0.37) (47) {};
    \node[circle,fill,inner sep=1pt] at (4.77, 3.51) (48) {};
    \node[circle,fill,inner sep=1pt] at (3.48, 3.59) (49) {};
    \node[circle,fill,inner sep=1pt] at (3.42, 2.07) (50) {};
    \node[circle,fill,inner sep=1pt] at (7.70, 4.17) (51) {};
    \node[circle,fill,inner sep=1pt] at (10.00, 1.41) (52) {};

    \draw (1) -- (49);
    \draw (49) -- (32);
    \draw (32) -- (45);
    \draw (45) -- (19);
    \draw (19) -- (41);
    \draw (41) -- (8);
    \draw (8) -- (9);
    \draw (9) -- (10);
    \draw (10) -- (43);
    \draw (43) -- (33);
    \draw (33) -- (51);
    \draw (51) -- (11);
    \draw (11) -- (52);
    \draw (52) -- (14);
    \draw (14) -- (13);
    \draw (13) -- (47);
    \draw (47) -- (26);
    \draw (26) -- (27);
    \draw (27) -- (28);
    \draw (28) -- (12);
    \draw (12) -- (25);
    \draw (25) -- (4);
    \draw (4) -- (6);
    \draw (6) -- (15);
    \draw (15) -- (5);
    \draw (5) -- (24);
    \draw (24) -- (48);
    \draw (48) -- (38);
    \draw (38) -- (37);
    \draw (37) -- (40);
    \draw (40) -- (39);
    \draw (39) -- (36);
    \draw (36) -- (35);
    \draw (35) -- (34);
    \draw (34) -- (44);
    \draw (44) -- (46);
    \draw (46) -- (16);
    \draw (16) -- (29);
    \draw (29) -- (50);
    \draw (50) -- (20);
    \draw (20) -- (23);
    \draw (23) -- (30);
    \draw (30) -- (2);
    \draw (2) -- (7);
    \draw (7) -- (42);
    \draw (42) -- (21);
    \draw (21) -- (17);
    \draw (17) -- (3);
    \draw (3) -- (18);
    \draw (18) -- (31);
    \draw (31) -- (22);
    \draw[densely dotted] (22) -- (1);
  \end{tikzpicture}
\end{figure}

\begin{figure}[H]
  \caption{Chemin tracé par l'heuristique \og Voisin le plus proche \fg}
  \center
  \begin{tikzpicture}
    \node[circle,fill,inner sep=1pt] at (3.25, 3.30) (1) {};
    \node[circle,fill,inner sep=1pt] at (0.14, 1.06) (2) {};
    \node[circle,fill,inner sep=1pt] at (1.98, 4.31) (3) {};
    \node[circle,fill,inner sep=1pt] at (5.43, 3.94) (4) {};
    \node[circle,fill,inner sep=1pt] at (4.86, 3.76) (5) {};
    \node[circle,fill,inner sep=1pt] at (5.06, 3.79) (6) {};
    \node[circle,fill,inner sep=1pt] at (0.14, 1.32) (7) {};
    \node[circle,fill,inner sep=1pt] at (3.02, 5.75) (8) {};
    \node[circle,fill,inner sep=1pt] at (3.33, 6.75) (9) {};
    \node[circle,fill,inner sep=1pt] at (3.74, 6.49) (10) {};
    \node[circle,fill,inner sep=1pt] at (9.22, 3.56) (11) {};
    \node[circle,fill,inner sep=1pt] at (7.01, 3.33) (12) {};
    \node[circle,fill,inner sep=1pt] at (8.42, 1.15) (13) {};
    \node[circle,fill,inner sep=1pt] at (8.79, 0.03) (14) {};
    \node[circle,fill,inner sep=1pt] at (4.86, 3.91) (15) {};
    \node[circle,fill,inner sep=1pt] at (4.17, 2.13) (16) {};
    \node[circle,fill,inner sep=1pt] at (0.83, 3.82) (17) {};
    \node[circle,fill,inner sep=1pt] at (2.39, 3.65) (18) {};
    \node[circle,fill,inner sep=1pt] at (2.93, 5.03) (19) {};
    \node[circle,fill,inner sep=1pt] at (3.22, 2.10) (20) {};
    \node[circle,fill,inner sep=1pt] at (1.72, 2.67) (21) {};
    \node[circle,fill,inner sep=1pt] at (2.99, 3.36) (22) {};
    \node[circle,fill,inner sep=1pt] at (2.76, 2.39) (23) {};
    \node[circle,fill,inner sep=1pt] at (4.80, 3.59) (24) {};
    \node[circle,fill,inner sep=1pt] at (5.60, 3.33) (25) {};
    \node[circle,fill,inner sep=1pt] at (6.98, 1.41) (26) {};
    \node[circle,fill,inner sep=1pt] at (7.59, 1.81) (27) {};
    \node[circle,fill,inner sep=1pt] at (7.18, 2.30) (28) {};
    \node[circle,fill,inner sep=1pt] at (3.79, 1.03) (29) {};
    \node[circle,fill,inner sep=1pt] at (2.36, 1.44) (30) {};
    \node[circle,fill,inner sep=1pt] at (2.41, 3.19) (31) {};
    \node[circle,fill,inner sep=1pt] at (3.30, 3.82) (32) {};
    \node[circle,fill,inner sep=1pt] at (6.61, 6.67) (33) {};
    \node[circle,fill,inner sep=1pt] at (4.02, 3.33) (34) {};
    \node[circle,fill,inner sep=1pt] at (3.94, 3.42) (35) {};
    \node[circle,fill,inner sep=1pt] at (3.94, 3.51) (36) {};
    \node[circle,fill,inner sep=1pt] at (4.43, 3.51) (37) {};
    \node[circle,fill,inner sep=1pt] at (4.57, 3.71) (38) {};
    \node[circle,fill,inner sep=1pt] at (4.14, 3.65) (39) {};
    \node[circle,fill,inner sep=1pt] at (4.37, 3.74) (40) {};
    \node[circle,fill,inner sep=1pt] at (2.73, 5.52) (41) {};
    \node[circle,fill,inner sep=1pt] at (0.55, 1.49) (42) {};
    \node[circle,fill,inner sep=1pt] at (5.03, 5.29) (43) {};
    \node[circle,fill,inner sep=1pt] at (4.02, 2.87) (44) {};
    \node[circle,fill,inner sep=1pt] at (3.19, 4.68) (45) {};
    \node[circle,fill,inner sep=1pt] at (4.77, 2.79) (46) {};
    \node[circle,fill,inner sep=1pt] at (6.72, 0.37) (47) {};
    \node[circle,fill,inner sep=1pt] at (4.77, 3.51) (48) {};
    \node[circle,fill,inner sep=1pt] at (3.48, 3.59) (49) {};
    \node[circle,fill,inner sep=1pt] at (3.42, 2.07) (50) {};
    \node[circle,fill,inner sep=1pt] at (7.70, 4.17) (51) {};
    \node[circle,fill,inner sep=1pt] at (10.00, 1.41) (52) {};

    \draw[darkcyan] (1) -- (22);
    \draw[darkcyan] (22) -- (49);
    \draw[darkcyan] (49) -- (32);
    \draw[darkcyan] (32) -- (36);
    \draw[darkcyan] (36) -- (35);
    \draw[darkcyan] (35) -- (34);
    \draw[darkcyan] (34) -- (39);
    \draw[darkcyan] (39) -- (40);
    \draw[darkcyan] (40) -- (38);
    \draw[darkcyan] (38) -- (37);
    \draw[darkcyan] (37) -- (48);
    \draw[darkcyan] (48) -- (24);
    \draw[darkcyan] (24) -- (5);
    \draw[darkcyan] (5) -- (15);
    \draw[darkcyan] (15) -- (6);
    \draw[darkcyan] (6) -- (4);
    \draw[darkcyan] (4) -- (25);
    \draw[darkcyan] (25) -- (46);
    \draw[darkcyan] (46) -- (44);
    \draw[darkcyan] (44) -- (16);
    \draw[darkcyan] (16) -- (50);
    \draw[darkcyan] (50) -- (20);
    \draw[darkcyan] (20) -- (23);
    \draw[darkcyan] (23) -- (31);
    \draw[darkcyan] (31) -- (18);
    \draw[darkcyan] (18) -- (3);
    \draw[darkcyan] (3) -- (19);
    \draw[darkcyan] (19) -- (45);
    \draw[darkcyan] (45) -- (41);
    \draw[darkcyan] (41) -- (8);
    \draw[darkcyan] (8) -- (10);
    \draw[darkcyan] (10) -- (9);
    \draw[darkcyan] (9) -- (43);
    \draw[darkcyan] (43) -- (33);
    \draw[darkcyan] (33) -- (51);
    \draw[darkcyan] (51) -- (12);
    \draw[darkcyan] (12) -- (28);
    \draw[darkcyan] (28) -- (27);
    \draw[darkcyan] (27) -- (26);
    \draw[darkcyan] (26) -- (47);
    \draw[darkcyan] (47) -- (13);
    \draw[darkcyan] (13) -- (14);
    \draw[darkcyan] (14) -- (52);
    \draw[darkcyan] (52) -- (11);
    \draw[darkcyan] (11) -- (29);
    \draw[darkcyan] (29) -- (30);
    \draw[darkcyan] (30) -- (21);
    \draw[darkcyan] (21) -- (17);
    \draw[darkcyan] (17) -- (42);
    \draw[darkcyan] (42) -- (7);
    \draw[darkcyan] (7) -- (2);
    \draw[darkcyan,densely dotted] (2) -- (1);
  \end{tikzpicture}
\end{figure}


\subsection{Mesure des performances}

L'algorithme de Little ne parvient pas à résoudre le problème pour $N_{cities} \geq 24$ en moins d'une heure.
Pour $N_{cities} = 23$, l'algorithme de Little prend $31'\,14''$ et pour $N_{cities} = 24$, il prend $3:46'$.

Le programme GLPK, quant à lui, ne parvient pas à résoudre le problème pour $N_{cities} \geq 21$ en moins d'une heure.
On observe qu'il prend tout de même seulement $40''$ pour $N_{cities} = 20$, mais plus d'une heure $N_{cities} = 21$.

L'algorithme de Little$^{+}$ performe le mieux de tous les trois, ce qui se comprend par sa capacité à éliminer tôt un grand nombre de branches invalides.

\begin{figure}[H]
  \caption{Temps pris par Little (linéaire/logarithmique)}
  \begin{tikzpicture}
    \begin{axis}[
      ybar,
      bar width=0.4cm,
      width=\columnwidth,
      height=0.5\columnwidth,
      ymax=1000,
      xtick=data,
      ymajorgrids=true,
      xlabel={$N_{cities}$},
      xtick align=inside,
      ylabel={Temps (s, lin)}
    ]
      \addplot+[ybar,fill=orange,color=orange,thick] table[col sep=comma, header=true, x index=0, y index=1] {little.csv};
    \end{axis}
  \end{tikzpicture}

  \begin{tikzpicture}
    \begin{axis}[
      width=\columnwidth,
      height=0.5\columnwidth,
      ymode=log,
      log ticks with fixed point,
      ymin=0.0001,
      ymax=1000,
      ymajorgrids=true,
      xtick=data,
      xlabel={$N_{cities}$},
      ylabel={Temps (s, log)}
    ]
      \addplot+[color=orange,mark options={fill=orange},thick] table[col sep=comma, header=true, x index=0, y index=1] {little.csv};
    \end{axis}
  \end{tikzpicture}
\end{figure}

% echo '"n", "Time (s)"' | tee glpk.csv; for n in `seq 3 52`; do sed -i -E "s/param n := [0-9]+/param n := ${n}/" tsp.mod; T="$({\time -f "%e" glpsol -m tsp.mod} 2>&1 1>&3 3>&-)" 3>/dev/null; echo "$n, $T" | tee -a glpk.csv; done

\begin{figure}[H]
  \caption{Temps pris par GLPK (linéaire/logarithmique)}
  \begin{tikzpicture}
    \begin{axis}[
      ybar,
      bar width=0.4cm,
      width=\columnwidth,
      height=0.5\columnwidth,
      ymax=1000,
      xtick=data,
      ymajorgrids=true,
      xlabel={$N_{cities}$},
      xtick align=inside,
      ylabel={Temps (s, lin)}
    ]
      \addplot+[ybar,color=darkcyan,fill=darkcyan,thick] table[col sep=comma, header=true, x index=0, y index=1] {glpk.csv};
    \end{axis}
  \end{tikzpicture}

  \begin{tikzpicture}
    \begin{axis}[
      width=\columnwidth,
      height=0.5\columnwidth,
      ymode=log,
      log ticks with fixed point,
      ymin=0.01,
      ymax=1000,
      ymajorgrids=true,
      xtick=data,
      xlabel={$N_{cities}$},
      ylabel={Temps (s, log)}
    ]
      \addplot+[color=darkcyan,mark options={fill=darkcyan},thick] table[col sep=comma, header=true, x index=0, y index=1] {glpk.csv};
    \end{axis}
  \end{tikzpicture}
\end{figure}

\begin{figure}[H]
  \caption{
    Temps pris par Little$^{+}$ (linéaire/logarithmique, comparaison entre les niveaux d'optimisation GCC) \\
    \textit{\texttt{-O0} désactive les optimisations et \texttt{-O3} optimise au maximum}
  }
  \begin{tikzpicture}
    \begin{axis}[
      ybar=0pt,
      % x=8pt,
      bar width=2pt,
      % bar shift=1,
      width=\columnwidth,
      height=0.5\columnwidth,
      ymax=60,
      xmin=5,
      xmax=53,
      % xtick=data,
      ymajorgrids=true,
      xlabel={$N_{cities}$},
      xtick align=inside,
      ylabel={Temps (s, lin)}
    ]
      \addplot+[ybar,color=purple,fill=purple,thick] table[col sep=comma, header=true, x index=0, y index=1] {littleplus.csv};
      \addplot+[ybar,color=purple,fill=white,thick] table[col sep=comma, header=true, x index=0, y index=1] {littleplus-o0.csv};
      \legend{\texttt{-O3},\texttt{-O0}}
    \end{axis}
  \end{tikzpicture}

  \begin{tikzpicture}
    \begin{axis}[
      width=0.958\columnwidth,
      height=0.5\columnwidth,
      ymode=log,
      log ticks with fixed point,
      ymin=0.0001,
      ymax=250,
      ymajorgrids=true,
      xmin=5,
      xmax=53,
      % xtick=data,
      xlabel={$N_{cities}$},
      ylabel={Temps (s, log)},
      legend pos=south east
    ]
      \addplot+[color=purple,mark=*,mark options={fill=purple},thick] table[col sep=comma, header=true, x index=0, y index=1] {littleplus.csv};
      \addplot+[color=purple,mark=o,mark options={fill=purple},thick] table[col sep=comma, header=true, x index=0, y index=1] {littleplus-o0.csv};
      \legend{\texttt{-O3},\texttt{-O0}}
    \end{axis}
  \end{tikzpicture}
\end{figure}

\begin{figure}[H]
  \caption{Temps combiné (linéaire/logarithmique)}
  \begin{tikzpicture}
    \begin{axis}[
      ybar,
      bar width=2pt,
      width=\columnwidth,
      height=0.5\columnwidth,
      ymax=1000,
      ymin=0,
      % xtick=data,
      ymajorgrids=true,
      xlabel={$N_{cities}$},
      xmin=5,
      xmax=53,
      xtick align=inside,
      ylabel={Temps (s, lin)},
      legend style={legend columns=-1}
    ]
      \addplot+[ybar,color=purple,fill=purple,thick] table[col sep=comma, header=true, x index=0, y index=1] {littleplus.csv};
      \addplot+[ybar,color=orange,fill=orange,thick] table[col sep=comma, header=true, x index=0, y index=1] {little.csv};
      \addplot+[ybar,color=darkcyan,fill=darkcyan,thick] table[col sep=comma, header=true, x index=0, y index=1] {glpk.csv};
      \legend{Little$^{+}$,Little,GLPK}
    \end{axis}
  \end{tikzpicture}

  \begin{tikzpicture}
    \begin{axis}[
      width=0.958\columnwidth,
      height=0.5\columnwidth,
      ymode=log,
      log ticks with fixed point,
      ymin=0.0001,
      ymax=1000,
      ymajorgrids=true,
      % xtick=data,
      xlabel={$N_{cities}$},
      xmin=5,
      xmax=53,
      ylabel={Temps (s, log)},
      legend style={legend columns=-1}
    ]
      \addplot+[color=purple,mark=*,mark options={fill=purple},thick] table[col sep=comma, header=true, x index=0, y index=1] {littleplus.csv};
      \addplot+[color=orange,mark=o,mark options={fill=orange},thick] table[col sep=comma, header=true, x index=0, y index=1] {little.csv};
      \addplot+[color=darkcyan,mark=square*,mark options={fill=darkcyan},thick] table[col sep=comma, header=true, x index=0, y index=1] {glpk.csv};
      \legend{Little$^{+}$,Little,GLPK}
    \end{axis}
  \end{tikzpicture}
\end{figure}

\end{document}
